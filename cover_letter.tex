\documentclass{article}

\usepackage{url}

\begin{document}

We are very grateful to the reviewers for having accepted to review our 
work, for the care they took in examining it and their resulting 
comments which have been appreciated in order to improve the quality of
the paper.

In the following three sections, we try to address separately each of
the concerns of our three reviewers.

\section{Comments of the first reviewer}

\subsection{How robust is the approach with regard to perceptual recognition
errors?}

We are aware that perception is not robust and therefore one of our next steps
is to include a probabilistic layer to the approach (as already mentioned in
the discussion section). We believe though that a robot (specially interacting
with people) should be allowed to make mistakes due to perception noise, or not
to be sure about something (for instance, "I think that..."), which represents
the certainty the robot has about a fact. In the meantime, to overcome
perception errors, the robot provides visual feedback (looking at objects) to
the user when possible and/or verifies the information with the user whenever
is needed (for instance, "do you mean the box\_01"?). 

\subsection{Does the cognitive model enable a meaningful interaction with a
real user in a real-world setting?}

Due to the perception problems and computational delays, fully realistic
settings are not yet possible.  The interaction based only on a static setting,
i.e. objects don't move, the interaction with a real user is accurate enough.
We have developed some experiments, where the user interacts with the system
only through a keyboard (no gestures are included) asking questions, providing
information, etc. and the outcome is quite promising. However, for dynamic
settings where the user can point at things, move the objects around, etc., the
interaction is not fast enough and the user should perform slow movements for
the robot to create a consistent state of world.

At recent video showing the current state of the interaction can be viewed
here: \url{http://www.youtube.com/watch?v=IODx50uV_k4}

\subsection{Rework of the introduction} The paper starts off with a discussion
of Austins speech act theory, including locutionary act, illocutionary force
and perlocutionary act. As the concept of speech acts is not really picked up
in the remainder of the paper, I do not see the point in starting with such a
discussion. It would be more appropriate to motivate how a robot could benefit
from a symbolic knowledge base etc.

I very strongly recommend that section I.A. be completely revised.


\section{Comments of the second reviewer}

\begin{itemize}
\item All reviewer comments regarding citations have been addressed, as suggested by the reviewer,
\item The reviewer's remark on human tendency to explain in a consistent way the world has been added
\item Language issues have been addressed as well.
\end{itemize}

\section{Comments of the third reviewer}

\subsection{``Explain the relation of the presented approach to the Open World
Assumption''}


\subsection{``Present a systematic overview of the formalisms and
representations used (OWL, RDF, DL, ...), and choose one consistent syntax for
listing the examples''}

The paragraph 2.2 has been reorganized to clarify the vocabulary and better
define the various acronyms.

\subsection{``The authors are encouraged to explain the notion of curiosity
with respect to the Dora robot by Hawes et al. (2011, ICRA) and Hanheide et al.
(2011, IJCAI)''}

\subsection{``Give examples of the alignment of ORO with OpenCyc''}

\subsection{Typography and notation consistency}

Typography has been throughly revised for better consistency. Notation varients have been
made explicit (for instance, section 2.4).

\subsection{``p.6, II.E. explain "multi-lingual support"''}


\subsection{``Points At: please describe how you manage the critical aspect of
timing''}

PointsAt relation is computed continuously during the interaction by 
selecting the objects that "might" be pointed at by the human. The outcome
of the computation (facts like "human1 pointsAt bottle1") are then
continously updated in the knowledge base.

In the current system, the decision on whether the human is pointing at an 
object in a given time is left to the reasoning on the dialog or on the human gestures:
when the dialogue module analyses a sentence like "give me this", it tries to
resolve the atom "this" by querying the ontology.

Both sitatuation assessement and dialogue processing chains induce 
small delays, but these are roughly the same (below 200ms), and 
compensate for each other.

\subsection{``p.8, III.B.Location according to an object: the authors are
kindly referred to the work done by, e.g., K. Sjöö or J. Kelleher''}

\subsection{``p.8, III.B. "creation of imaginary objects...hinted by verbal
assertions..." -- this phenomenon is called presupposition accommodation''}

\subsection{``p.9, Alg. IV.1: the crucial bits are left out; please explain
"GenerateDescription", "Ontology.Find", "Ontology.CheckEquivalent", and
"Discrimination"''}

\subsection{``how does the system deal with restrictive information vs.
attributive information expressed by adjectives (e.g., in "the yellow banana is
big")?''}

\subsection{``the authors are kindly referred to the work by Zender et al.
(2009, IJCAI) on reference resolution in OWL-DL ontology knowledge bases using
SPARQL queries''}

\subsection{``VI. the authors are advised to present an initial overview of the
robot hardware and the common software architecture used in the systems''}

\subsection{``VI. which university in Munich? LMU or TUM?''}

\subsection{``VI. please present the anecdote of unexpected behavior elsewhere
or give some more context''}

\subsection{``Fig. 15. please make the spelling and the presentation more
consistent''}

\subsection{``the authors are kindly referred to work on the Generation and
Resolution of Referring Expressions (GRE, RRE) from the Computational
Linguistics and Natural Language Processing community, e.g. Dale\&Reiter,
Krahmer\&Theune, Kelleher ''}

\subsection{``please improve the quality of the references (sometimes, the
publisher is missing, sometimes capitalization is omitted, sometimes
abbreviations are used and sometimes full names''}

\end{document}
